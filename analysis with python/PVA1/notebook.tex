
% Default to the notebook output style

    


% Inherit from the specified cell style.




    
\documentclass[11pt]{article}

    
    
    \usepackage[T1]{fontenc}
    % Nicer default font (+ math font) than Computer Modern for most use cases
    \usepackage{mathpazo}

    % Basic figure setup, for now with no caption control since it's done
    % automatically by Pandoc (which extracts ![](path) syntax from Markdown).
    \usepackage{graphicx}
    % We will generate all images so they have a width \maxwidth. This means
    % that they will get their normal width if they fit onto the page, but
    % are scaled down if they would overflow the margins.
    \makeatletter
    \def\maxwidth{\ifdim\Gin@nat@width>\linewidth\linewidth
    \else\Gin@nat@width\fi}
    \makeatother
    \let\Oldincludegraphics\includegraphics
    % Set max figure width to be 80% of text width, for now hardcoded.
    \renewcommand{\includegraphics}[1]{\Oldincludegraphics[width=.8\maxwidth]{#1}}
    % Ensure that by default, figures have no caption (until we provide a
    % proper Figure object with a Caption API and a way to capture that
    % in the conversion process - todo).
    \usepackage{caption}
    \DeclareCaptionLabelFormat{nolabel}{}
    \captionsetup{labelformat=nolabel}

    \usepackage{adjustbox} % Used to constrain images to a maximum size 
    \usepackage{xcolor} % Allow colors to be defined
    \usepackage{enumerate} % Needed for markdown enumerations to work
    \usepackage{geometry} % Used to adjust the document margins
    \usepackage{amsmath} % Equations
    \usepackage{amssymb} % Equations
    \usepackage{textcomp} % defines textquotesingle
    % Hack from http://tex.stackexchange.com/a/47451/13684:
    \AtBeginDocument{%
        \def\PYZsq{\textquotesingle}% Upright quotes in Pygmentized code
    }
    \usepackage{upquote} % Upright quotes for verbatim code
    \usepackage{eurosym} % defines \euro
    \usepackage[mathletters]{ucs} % Extended unicode (utf-8) support
    \usepackage[utf8x]{inputenc} % Allow utf-8 characters in the tex document
    \usepackage{fancyvrb} % verbatim replacement that allows latex
    \usepackage{grffile} % extends the file name processing of package graphics 
                         % to support a larger range 
    % The hyperref package gives us a pdf with properly built
    % internal navigation ('pdf bookmarks' for the table of contents,
    % internal cross-reference links, web links for URLs, etc.)
    \usepackage{hyperref}
    \usepackage{longtable} % longtable support required by pandoc >1.10
    \usepackage{booktabs}  % table support for pandoc > 1.12.2
    \usepackage[inline]{enumitem} % IRkernel/repr support (it uses the enumerate* environment)
    \usepackage[normalem]{ulem} % ulem is needed to support strikethroughs (\sout)
                                % normalem makes italics be italics, not underlines
    

    
    
    % Colors for the hyperref package
    \definecolor{urlcolor}{rgb}{0,.145,.698}
    \definecolor{linkcolor}{rgb}{.71,0.21,0.01}
    \definecolor{citecolor}{rgb}{.12,.54,.11}

    % ANSI colors
    \definecolor{ansi-black}{HTML}{3E424D}
    \definecolor{ansi-black-intense}{HTML}{282C36}
    \definecolor{ansi-red}{HTML}{E75C58}
    \definecolor{ansi-red-intense}{HTML}{B22B31}
    \definecolor{ansi-green}{HTML}{00A250}
    \definecolor{ansi-green-intense}{HTML}{007427}
    \definecolor{ansi-yellow}{HTML}{DDB62B}
    \definecolor{ansi-yellow-intense}{HTML}{B27D12}
    \definecolor{ansi-blue}{HTML}{208FFB}
    \definecolor{ansi-blue-intense}{HTML}{0065CA}
    \definecolor{ansi-magenta}{HTML}{D160C4}
    \definecolor{ansi-magenta-intense}{HTML}{A03196}
    \definecolor{ansi-cyan}{HTML}{60C6C8}
    \definecolor{ansi-cyan-intense}{HTML}{258F8F}
    \definecolor{ansi-white}{HTML}{C5C1B4}
    \definecolor{ansi-white-intense}{HTML}{A1A6B2}

    % commands and environments needed by pandoc snippets
    % extracted from the output of `pandoc -s`
    \providecommand{\tightlist}{%
      \setlength{\itemsep}{0pt}\setlength{\parskip}{0pt}}
    \DefineVerbatimEnvironment{Highlighting}{Verbatim}{commandchars=\\\{\}}
    % Add ',fontsize=\small' for more characters per line
    \newenvironment{Shaded}{}{}
    \newcommand{\KeywordTok}[1]{\textcolor[rgb]{0.00,0.44,0.13}{\textbf{{#1}}}}
    \newcommand{\DataTypeTok}[1]{\textcolor[rgb]{0.56,0.13,0.00}{{#1}}}
    \newcommand{\DecValTok}[1]{\textcolor[rgb]{0.25,0.63,0.44}{{#1}}}
    \newcommand{\BaseNTok}[1]{\textcolor[rgb]{0.25,0.63,0.44}{{#1}}}
    \newcommand{\FloatTok}[1]{\textcolor[rgb]{0.25,0.63,0.44}{{#1}}}
    \newcommand{\CharTok}[1]{\textcolor[rgb]{0.25,0.44,0.63}{{#1}}}
    \newcommand{\StringTok}[1]{\textcolor[rgb]{0.25,0.44,0.63}{{#1}}}
    \newcommand{\CommentTok}[1]{\textcolor[rgb]{0.38,0.63,0.69}{\textit{{#1}}}}
    \newcommand{\OtherTok}[1]{\textcolor[rgb]{0.00,0.44,0.13}{{#1}}}
    \newcommand{\AlertTok}[1]{\textcolor[rgb]{1.00,0.00,0.00}{\textbf{{#1}}}}
    \newcommand{\FunctionTok}[1]{\textcolor[rgb]{0.02,0.16,0.49}{{#1}}}
    \newcommand{\RegionMarkerTok}[1]{{#1}}
    \newcommand{\ErrorTok}[1]{\textcolor[rgb]{1.00,0.00,0.00}{\textbf{{#1}}}}
    \newcommand{\NormalTok}[1]{{#1}}
    
    % Additional commands for more recent versions of Pandoc
    \newcommand{\ConstantTok}[1]{\textcolor[rgb]{0.53,0.00,0.00}{{#1}}}
    \newcommand{\SpecialCharTok}[1]{\textcolor[rgb]{0.25,0.44,0.63}{{#1}}}
    \newcommand{\VerbatimStringTok}[1]{\textcolor[rgb]{0.25,0.44,0.63}{{#1}}}
    \newcommand{\SpecialStringTok}[1]{\textcolor[rgb]{0.73,0.40,0.53}{{#1}}}
    \newcommand{\ImportTok}[1]{{#1}}
    \newcommand{\DocumentationTok}[1]{\textcolor[rgb]{0.73,0.13,0.13}{\textit{{#1}}}}
    \newcommand{\AnnotationTok}[1]{\textcolor[rgb]{0.38,0.63,0.69}{\textbf{\textit{{#1}}}}}
    \newcommand{\CommentVarTok}[1]{\textcolor[rgb]{0.38,0.63,0.69}{\textbf{\textit{{#1}}}}}
    \newcommand{\VariableTok}[1]{\textcolor[rgb]{0.10,0.09,0.49}{{#1}}}
    \newcommand{\ControlFlowTok}[1]{\textcolor[rgb]{0.00,0.44,0.13}{\textbf{{#1}}}}
    \newcommand{\OperatorTok}[1]{\textcolor[rgb]{0.40,0.40,0.40}{{#1}}}
    \newcommand{\BuiltInTok}[1]{{#1}}
    \newcommand{\ExtensionTok}[1]{{#1}}
    \newcommand{\PreprocessorTok}[1]{\textcolor[rgb]{0.74,0.48,0.00}{{#1}}}
    \newcommand{\AttributeTok}[1]{\textcolor[rgb]{0.49,0.56,0.16}{{#1}}}
    \newcommand{\InformationTok}[1]{\textcolor[rgb]{0.38,0.63,0.69}{\textbf{\textit{{#1}}}}}
    \newcommand{\WarningTok}[1]{\textcolor[rgb]{0.38,0.63,0.69}{\textbf{\textit{{#1}}}}}
    
    
    % Define a nice break command that doesn't care if a line doesn't already
    % exist.
    \def\br{\hspace*{\fill} \\* }
    % Math Jax compatability definitions
    \def\gt{>}
    \def\lt{<}
    % Document parameters
    \title{PVA1\_Aufgaben\_David\_Marwik}
    
    
    

    % Pygments definitions
    
\makeatletter
\def\PY@reset{\let\PY@it=\relax \let\PY@bf=\relax%
    \let\PY@ul=\relax \let\PY@tc=\relax%
    \let\PY@bc=\relax \let\PY@ff=\relax}
\def\PY@tok#1{\csname PY@tok@#1\endcsname}
\def\PY@toks#1+{\ifx\relax#1\empty\else%
    \PY@tok{#1}\expandafter\PY@toks\fi}
\def\PY@do#1{\PY@bc{\PY@tc{\PY@ul{%
    \PY@it{\PY@bf{\PY@ff{#1}}}}}}}
\def\PY#1#2{\PY@reset\PY@toks#1+\relax+\PY@do{#2}}

\expandafter\def\csname PY@tok@w\endcsname{\def\PY@tc##1{\textcolor[rgb]{0.73,0.73,0.73}{##1}}}
\expandafter\def\csname PY@tok@c\endcsname{\let\PY@it=\textit\def\PY@tc##1{\textcolor[rgb]{0.25,0.50,0.50}{##1}}}
\expandafter\def\csname PY@tok@cp\endcsname{\def\PY@tc##1{\textcolor[rgb]{0.74,0.48,0.00}{##1}}}
\expandafter\def\csname PY@tok@k\endcsname{\let\PY@bf=\textbf\def\PY@tc##1{\textcolor[rgb]{0.00,0.50,0.00}{##1}}}
\expandafter\def\csname PY@tok@kp\endcsname{\def\PY@tc##1{\textcolor[rgb]{0.00,0.50,0.00}{##1}}}
\expandafter\def\csname PY@tok@kt\endcsname{\def\PY@tc##1{\textcolor[rgb]{0.69,0.00,0.25}{##1}}}
\expandafter\def\csname PY@tok@o\endcsname{\def\PY@tc##1{\textcolor[rgb]{0.40,0.40,0.40}{##1}}}
\expandafter\def\csname PY@tok@ow\endcsname{\let\PY@bf=\textbf\def\PY@tc##1{\textcolor[rgb]{0.67,0.13,1.00}{##1}}}
\expandafter\def\csname PY@tok@nb\endcsname{\def\PY@tc##1{\textcolor[rgb]{0.00,0.50,0.00}{##1}}}
\expandafter\def\csname PY@tok@nf\endcsname{\def\PY@tc##1{\textcolor[rgb]{0.00,0.00,1.00}{##1}}}
\expandafter\def\csname PY@tok@nc\endcsname{\let\PY@bf=\textbf\def\PY@tc##1{\textcolor[rgb]{0.00,0.00,1.00}{##1}}}
\expandafter\def\csname PY@tok@nn\endcsname{\let\PY@bf=\textbf\def\PY@tc##1{\textcolor[rgb]{0.00,0.00,1.00}{##1}}}
\expandafter\def\csname PY@tok@ne\endcsname{\let\PY@bf=\textbf\def\PY@tc##1{\textcolor[rgb]{0.82,0.25,0.23}{##1}}}
\expandafter\def\csname PY@tok@nv\endcsname{\def\PY@tc##1{\textcolor[rgb]{0.10,0.09,0.49}{##1}}}
\expandafter\def\csname PY@tok@no\endcsname{\def\PY@tc##1{\textcolor[rgb]{0.53,0.00,0.00}{##1}}}
\expandafter\def\csname PY@tok@nl\endcsname{\def\PY@tc##1{\textcolor[rgb]{0.63,0.63,0.00}{##1}}}
\expandafter\def\csname PY@tok@ni\endcsname{\let\PY@bf=\textbf\def\PY@tc##1{\textcolor[rgb]{0.60,0.60,0.60}{##1}}}
\expandafter\def\csname PY@tok@na\endcsname{\def\PY@tc##1{\textcolor[rgb]{0.49,0.56,0.16}{##1}}}
\expandafter\def\csname PY@tok@nt\endcsname{\let\PY@bf=\textbf\def\PY@tc##1{\textcolor[rgb]{0.00,0.50,0.00}{##1}}}
\expandafter\def\csname PY@tok@nd\endcsname{\def\PY@tc##1{\textcolor[rgb]{0.67,0.13,1.00}{##1}}}
\expandafter\def\csname PY@tok@s\endcsname{\def\PY@tc##1{\textcolor[rgb]{0.73,0.13,0.13}{##1}}}
\expandafter\def\csname PY@tok@sd\endcsname{\let\PY@it=\textit\def\PY@tc##1{\textcolor[rgb]{0.73,0.13,0.13}{##1}}}
\expandafter\def\csname PY@tok@si\endcsname{\let\PY@bf=\textbf\def\PY@tc##1{\textcolor[rgb]{0.73,0.40,0.53}{##1}}}
\expandafter\def\csname PY@tok@se\endcsname{\let\PY@bf=\textbf\def\PY@tc##1{\textcolor[rgb]{0.73,0.40,0.13}{##1}}}
\expandafter\def\csname PY@tok@sr\endcsname{\def\PY@tc##1{\textcolor[rgb]{0.73,0.40,0.53}{##1}}}
\expandafter\def\csname PY@tok@ss\endcsname{\def\PY@tc##1{\textcolor[rgb]{0.10,0.09,0.49}{##1}}}
\expandafter\def\csname PY@tok@sx\endcsname{\def\PY@tc##1{\textcolor[rgb]{0.00,0.50,0.00}{##1}}}
\expandafter\def\csname PY@tok@m\endcsname{\def\PY@tc##1{\textcolor[rgb]{0.40,0.40,0.40}{##1}}}
\expandafter\def\csname PY@tok@gh\endcsname{\let\PY@bf=\textbf\def\PY@tc##1{\textcolor[rgb]{0.00,0.00,0.50}{##1}}}
\expandafter\def\csname PY@tok@gu\endcsname{\let\PY@bf=\textbf\def\PY@tc##1{\textcolor[rgb]{0.50,0.00,0.50}{##1}}}
\expandafter\def\csname PY@tok@gd\endcsname{\def\PY@tc##1{\textcolor[rgb]{0.63,0.00,0.00}{##1}}}
\expandafter\def\csname PY@tok@gi\endcsname{\def\PY@tc##1{\textcolor[rgb]{0.00,0.63,0.00}{##1}}}
\expandafter\def\csname PY@tok@gr\endcsname{\def\PY@tc##1{\textcolor[rgb]{1.00,0.00,0.00}{##1}}}
\expandafter\def\csname PY@tok@ge\endcsname{\let\PY@it=\textit}
\expandafter\def\csname PY@tok@gs\endcsname{\let\PY@bf=\textbf}
\expandafter\def\csname PY@tok@gp\endcsname{\let\PY@bf=\textbf\def\PY@tc##1{\textcolor[rgb]{0.00,0.00,0.50}{##1}}}
\expandafter\def\csname PY@tok@go\endcsname{\def\PY@tc##1{\textcolor[rgb]{0.53,0.53,0.53}{##1}}}
\expandafter\def\csname PY@tok@gt\endcsname{\def\PY@tc##1{\textcolor[rgb]{0.00,0.27,0.87}{##1}}}
\expandafter\def\csname PY@tok@err\endcsname{\def\PY@bc##1{\setlength{\fboxsep}{0pt}\fcolorbox[rgb]{1.00,0.00,0.00}{1,1,1}{\strut ##1}}}
\expandafter\def\csname PY@tok@kc\endcsname{\let\PY@bf=\textbf\def\PY@tc##1{\textcolor[rgb]{0.00,0.50,0.00}{##1}}}
\expandafter\def\csname PY@tok@kd\endcsname{\let\PY@bf=\textbf\def\PY@tc##1{\textcolor[rgb]{0.00,0.50,0.00}{##1}}}
\expandafter\def\csname PY@tok@kn\endcsname{\let\PY@bf=\textbf\def\PY@tc##1{\textcolor[rgb]{0.00,0.50,0.00}{##1}}}
\expandafter\def\csname PY@tok@kr\endcsname{\let\PY@bf=\textbf\def\PY@tc##1{\textcolor[rgb]{0.00,0.50,0.00}{##1}}}
\expandafter\def\csname PY@tok@bp\endcsname{\def\PY@tc##1{\textcolor[rgb]{0.00,0.50,0.00}{##1}}}
\expandafter\def\csname PY@tok@fm\endcsname{\def\PY@tc##1{\textcolor[rgb]{0.00,0.00,1.00}{##1}}}
\expandafter\def\csname PY@tok@vc\endcsname{\def\PY@tc##1{\textcolor[rgb]{0.10,0.09,0.49}{##1}}}
\expandafter\def\csname PY@tok@vg\endcsname{\def\PY@tc##1{\textcolor[rgb]{0.10,0.09,0.49}{##1}}}
\expandafter\def\csname PY@tok@vi\endcsname{\def\PY@tc##1{\textcolor[rgb]{0.10,0.09,0.49}{##1}}}
\expandafter\def\csname PY@tok@vm\endcsname{\def\PY@tc##1{\textcolor[rgb]{0.10,0.09,0.49}{##1}}}
\expandafter\def\csname PY@tok@sa\endcsname{\def\PY@tc##1{\textcolor[rgb]{0.73,0.13,0.13}{##1}}}
\expandafter\def\csname PY@tok@sb\endcsname{\def\PY@tc##1{\textcolor[rgb]{0.73,0.13,0.13}{##1}}}
\expandafter\def\csname PY@tok@sc\endcsname{\def\PY@tc##1{\textcolor[rgb]{0.73,0.13,0.13}{##1}}}
\expandafter\def\csname PY@tok@dl\endcsname{\def\PY@tc##1{\textcolor[rgb]{0.73,0.13,0.13}{##1}}}
\expandafter\def\csname PY@tok@s2\endcsname{\def\PY@tc##1{\textcolor[rgb]{0.73,0.13,0.13}{##1}}}
\expandafter\def\csname PY@tok@sh\endcsname{\def\PY@tc##1{\textcolor[rgb]{0.73,0.13,0.13}{##1}}}
\expandafter\def\csname PY@tok@s1\endcsname{\def\PY@tc##1{\textcolor[rgb]{0.73,0.13,0.13}{##1}}}
\expandafter\def\csname PY@tok@mb\endcsname{\def\PY@tc##1{\textcolor[rgb]{0.40,0.40,0.40}{##1}}}
\expandafter\def\csname PY@tok@mf\endcsname{\def\PY@tc##1{\textcolor[rgb]{0.40,0.40,0.40}{##1}}}
\expandafter\def\csname PY@tok@mh\endcsname{\def\PY@tc##1{\textcolor[rgb]{0.40,0.40,0.40}{##1}}}
\expandafter\def\csname PY@tok@mi\endcsname{\def\PY@tc##1{\textcolor[rgb]{0.40,0.40,0.40}{##1}}}
\expandafter\def\csname PY@tok@il\endcsname{\def\PY@tc##1{\textcolor[rgb]{0.40,0.40,0.40}{##1}}}
\expandafter\def\csname PY@tok@mo\endcsname{\def\PY@tc##1{\textcolor[rgb]{0.40,0.40,0.40}{##1}}}
\expandafter\def\csname PY@tok@ch\endcsname{\let\PY@it=\textit\def\PY@tc##1{\textcolor[rgb]{0.25,0.50,0.50}{##1}}}
\expandafter\def\csname PY@tok@cm\endcsname{\let\PY@it=\textit\def\PY@tc##1{\textcolor[rgb]{0.25,0.50,0.50}{##1}}}
\expandafter\def\csname PY@tok@cpf\endcsname{\let\PY@it=\textit\def\PY@tc##1{\textcolor[rgb]{0.25,0.50,0.50}{##1}}}
\expandafter\def\csname PY@tok@c1\endcsname{\let\PY@it=\textit\def\PY@tc##1{\textcolor[rgb]{0.25,0.50,0.50}{##1}}}
\expandafter\def\csname PY@tok@cs\endcsname{\let\PY@it=\textit\def\PY@tc##1{\textcolor[rgb]{0.25,0.50,0.50}{##1}}}

\def\PYZbs{\char`\\}
\def\PYZus{\char`\_}
\def\PYZob{\char`\{}
\def\PYZcb{\char`\}}
\def\PYZca{\char`\^}
\def\PYZam{\char`\&}
\def\PYZlt{\char`\<}
\def\PYZgt{\char`\>}
\def\PYZsh{\char`\#}
\def\PYZpc{\char`\%}
\def\PYZdl{\char`\$}
\def\PYZhy{\char`\-}
\def\PYZsq{\char`\'}
\def\PYZdq{\char`\"}
\def\PYZti{\char`\~}
% for compatibility with earlier versions
\def\PYZat{@}
\def\PYZlb{[}
\def\PYZrb{]}
\makeatother


    % Exact colors from NB
    \definecolor{incolor}{rgb}{0.0, 0.0, 0.5}
    \definecolor{outcolor}{rgb}{0.545, 0.0, 0.0}



    
    % Prevent overflowing lines due to hard-to-break entities
    \sloppy 
    % Setup hyperref package
    \hypersetup{
      breaklinks=true,  % so long urls are correctly broken across lines
      colorlinks=true,
      urlcolor=urlcolor,
      linkcolor=linkcolor,
      citecolor=citecolor,
      }
    % Slightly bigger margins than the latex defaults
    
    \geometry{verbose,tmargin=1in,bmargin=1in,lmargin=1in,rmargin=1in}
    
    

    \begin{document}
    
    
    \maketitle
    
    

    
    \section{Semesterarbeit Analyis mit Python Teil
1}\label{semesterarbeit-analyis-mit-python-teil-1}

\subsection{Einleitung: Rekursion mit Python anhand der
Fibonacci-Folge}\label{einleitung-rekursion-mit-python-anhand-der-fibonacci-folge}

Zu ehren Leonardo Fibonaccis welcher im Jahr 1202 wohl kaum dachte
welche Signifikants seine Beschreibung zum Wachstum einer
Kanichenpopulation erreichen würde. Die Fibonacci-Folge wurde von ihm
wie folgt definiert:

\$ f\_0:=0 \textbackslash{} f\_1:=1 \textbackslash{}
f\_n:=f\_\{n−1\}+f\_\{n−2\} \text{ für } n≥2 \textbackslash{} \$

Die ersten Fibonacci-Zahlen sind folglich 0, 1, 1, 2, 3, 5, 8, 13, 21,
34, 55,...

Die Aufgabenstellung gliedert sich in fünf Teile:

\begin{enumerate}
\def\labelenumi{\arabic{enumi}.}
\tightlist
\item
  Implementieren Sie eine Python-Funktion fib(n) , die die n-te
  Fibonacci-Zahl bestimmt.
\item
  Eine naive Implementierung setzt die obige Rekursionsgleichung direkt
  um. Schreiben Sie eine weitere Python-Funktion, die berechnet, wie
  viele Funktionsaufrufe on fib notwendig sind, um die n-te
  Fibonacci-Zahl zu berechnen.
\item
  Vergleichen Sie die Anzahl der Funktionsaufrufe von fib zur Bestimmung
  einer Fibonacci-Zahl mit den Fibonacci-Zahlen selber. Können Sie eine
  Vermutung aufstellen?
\item
  Verwenden Sie die Funktion time() aus dem Modul time, um zu bestimmen,
  wie lange die Funktion fib benötigt, um eine Fibonacci-Zahl zu
  bestimmen.
\item
  Implementieren Sie eine weitere Python-Funktion zur Berechnung der
  n-ten Fibonacci-Zahl, die möglichst effizient ist. (Hinweis: das kann
  rekursiv oder iterativ gelöst werden.)
\end{enumerate}

\subsection{Theoretische Beschreibung des
Lösungsansatzes}\label{theoretische-beschreibung-des-luxf6sungsansatzes}

Die erste Aufgabe ist gelöst in dem die Definition der Fibonacci-Folge
sozusagen in die Phytonsprache übersetzt wird. Es werden drei
verscheidene Fälle abgefangen in der Funktion fib() für \(f_0\), \(f_1\)
und \(f_n\). Die Fälle haben dann den Rückgabewert gemäss Definition.

Für die zweite Aufgabe fügen wir obiger fib() Funktion noch einen Zähler
hinzu welcher bei jedem Aufruf erhöht wird.

In Aufgabe drei um die Anzahl Funktionsaufrufe mit den Fibonacci-Zahlen
selber zu untersuchen geben wir mit einer for-Schleife einemal die
Position, Fibonacci-Zahl und Anzahl Funktionsaufrufe der ersten 20
Zahlen aus. Somit können die Zahlen beurteilt werden. Danach lässt sich
eventuell ein Folge finden für die Anzahl Funktionsaufrufe. Diese wäre
mit dem Induktionsalgorithmus auf Richtigkeit zu prüfen.

Die Aufgabe 4. und 5. wird zusammen gefasst. Die Funktion fib() wird
sauber neu aufgebaut, wie ich das professionel tun würde. Dazu erstellen
wir eine Klasse in welcher wir die die Fibonacci-Zahlen iterativ und
rekursiv berechnen und dabei die Zeit messen. Danach wollen wir die
beiden Funktionen vergelichen und die schnellere bestimmen.

Die Itearative Funktion lässt sich mit einer Schleife realisieren welche
so oft die beiden Vorgänger zusammen zählt und die
\$f\_\{n−1\}+f\_\{n−2\} £, bis die gewünschte n-te Zahl erreicht ist.

\subsection{Implementierungsidee und Programm
Code}\label{implementierungsidee-und-programm-code}

\subsubsection{1. n-te Fibonacci-Zahl
bestimmen}\label{n-te-fibonacci-zahl-bestimmen}

Die Funktion fib() fängt die beiden Startwerte a\_0 und a\_1 mit einem
"if" und "elif" ab. Als letzte Option geht die Funktions ins "else" und
ruft sich selbst auf. Die Funktion ruft sich solange selbst auf bis alle
Selbstaufrufe bei 0 und 1 enden.

    \begin{Verbatim}[commandchars=\\\{\}]
{\color{incolor}In [{\color{incolor}1}]:} \PY{c+c1}{\PYZsh{}set the n\PYZhy{}the fibonaccinumber you would like to get.}
        \PY{n+nb}{print}\PY{p}{(}\PY{l+s+s2}{\PYZdq{}}\PY{l+s+se}{\PYZbs{}n}\PY{l+s+s2}{\PYZdq{}}\PY{p}{)}
        \PY{n+nb}{print}\PY{p}{(}\PY{l+s+s2}{\PYZdq{}}\PY{l+s+s2}{Please set the n\PYZhy{}the Fibonacci number you would like to get!}\PY{l+s+s2}{\PYZdq{}}\PY{p}{)}
        \PY{n}{n} \PY{o}{=} \PY{n+nb}{int}\PY{p}{(}\PY{n+nb}{input}\PY{p}{(}\PY{p}{)}\PY{p}{)}
        
        \PY{k}{def} \PY{n+nf}{fib}\PY{p}{(}\PY{n}{n}\PY{p}{)}\PY{p}{:}
            \PY{k}{if} \PY{n}{n} \PY{o}{==} \PY{l+m+mi}{0}\PY{p}{:}
                \PY{k}{return} \PY{l+m+mi}{0}
            \PY{k}{elif} \PY{n}{n} \PY{o}{==} \PY{l+m+mi}{1}\PY{p}{:}
                \PY{k}{return} \PY{l+m+mi}{1}
            \PY{k}{else}\PY{p}{:}
                \PY{k}{return} \PY{n}{fib}\PY{p}{(}\PY{n}{n}\PY{o}{\PYZhy{}}\PY{l+m+mi}{1}\PY{p}{)} \PY{o}{+} \PY{n}{fib}\PY{p}{(}\PY{n}{n}\PY{o}{\PYZhy{}}\PY{l+m+mi}{2}\PY{p}{)}
            
        \PY{n+nb}{print}\PY{p}{(}\PY{l+s+s2}{\PYZdq{}}\PY{l+s+se}{\PYZbs{}n}\PY{l+s+s2}{\PYZdq{}}\PY{p}{)}   
        \PY{n+nb}{print}\PY{p}{(}\PY{n}{f}\PY{l+s+s2}{\PYZdq{}}\PY{l+s+s2}{At position n=}\PY{l+s+si}{\PYZob{}n\PYZcb{}}\PY{l+s+s2}{ is Fibonacci number: }\PY{l+s+s2}{\PYZob{}}\PY{l+s+s2}{fib(n)\PYZcb{}}\PY{l+s+s2}{\PYZdq{}}\PY{p}{)}
\end{Verbatim}


    \begin{Verbatim}[commandchars=\\\{\}]


Please set the n-the Fibonacci number you would like to get!
3


At position n=3 is Fibonacci number: 2

    \end{Verbatim}

    \subsubsection{2. fib() Aufrufe}\label{fib-aufrufe}

Zur simplen Implementierung fügen wier einen Zähler hinzu welcher bei
jedem Funktionsaufruf um eins erhöht wird. Damit wissen wir wie oft
fib() aufgerufen wird. Dazu wird hier eine Variabel "count" eingeführt.

    \begin{Verbatim}[commandchars=\\\{\}]
{\color{incolor}In [{\color{incolor}2}]:} \PY{c+c1}{\PYZsh{}set the n\PYZhy{}the fibonaccinumber you would like to get.}
        \PY{n+nb}{print}\PY{p}{(}\PY{l+s+s2}{\PYZdq{}}\PY{l+s+se}{\PYZbs{}n}\PY{l+s+s2}{\PYZdq{}}\PY{p}{)}
        \PY{n+nb}{print}\PY{p}{(}\PY{l+s+s2}{\PYZdq{}}\PY{l+s+s2}{Please set the n\PYZhy{}the Fibonacci number you would like to get!}\PY{l+s+s2}{\PYZdq{}}\PY{p}{)}
        \PY{n}{n} \PY{o}{=} \PY{n+nb}{int}\PY{p}{(}\PY{n+nb}{input}\PY{p}{(}\PY{p}{)}\PY{p}{)}
        
        \PY{c+c1}{\PYZsh{}count of function calls}
        \PY{n}{count} \PY{o}{=} \PY{l+m+mi}{0}
        
        \PY{k}{def} \PY{n+nf}{fib}\PY{p}{(}\PY{n}{n}\PY{p}{)}\PY{p}{:}
            \PY{k}{global} \PY{n}{count}
            \PY{n}{count} \PY{o}{+}\PY{o}{=} \PY{l+m+mi}{1}
            
            \PY{k}{if} \PY{n}{n} \PY{o}{==} \PY{l+m+mi}{0}\PY{p}{:}
                \PY{k}{return} \PY{l+m+mi}{0}
            \PY{k}{elif} \PY{n}{n} \PY{o}{==} \PY{l+m+mi}{1}\PY{p}{:}        
                \PY{k}{return} \PY{l+m+mi}{1}
            \PY{k}{else}\PY{p}{:}        
                \PY{k}{return} \PY{n}{fib}\PY{p}{(}\PY{n}{n}\PY{o}{\PYZhy{}}\PY{l+m+mi}{1}\PY{p}{)} \PY{o}{+} \PY{n}{fib}\PY{p}{(}\PY{n}{n}\PY{o}{\PYZhy{}}\PY{l+m+mi}{2}\PY{p}{)}
            
        \PY{n+nb}{print}\PY{p}{(}\PY{l+s+s2}{\PYZdq{}}\PY{l+s+se}{\PYZbs{}n}\PY{l+s+s2}{\PYZdq{}}\PY{p}{)}   
        \PY{n+nb}{print}\PY{p}{(}\PY{n}{f}\PY{l+s+s2}{\PYZdq{}}\PY{l+s+s2}{At position n=}\PY{l+s+si}{\PYZob{}n\PYZcb{}}\PY{l+s+s2}{ is Fibonacci number: }\PY{l+s+s2}{\PYZob{}}\PY{l+s+s2}{fib(n)\PYZcb{}}\PY{l+s+s2}{\PYZdq{}}\PY{p}{)}
        \PY{n+nb}{print}\PY{p}{(}\PY{n}{f}\PY{l+s+s2}{\PYZdq{}}\PY{l+s+s2}{It took }\PY{l+s+si}{\PYZob{}count\PYZcb{}}\PY{l+s+s2}{ calls of fib()}\PY{l+s+s2}{\PYZdq{}}\PY{p}{)}
\end{Verbatim}


    \begin{Verbatim}[commandchars=\\\{\}]


Please set the n-the Fibonacci number you would like to get!
4


At position n=4 is Fibonacci number: 3
It took 9 calls of fib()

    \end{Verbatim}

    \subsection{3. Anzahl Funktionsaufrufe}\label{anzahl-funktionsaufrufe}

Die for-Schleife plotet die ersten 20 Fibonacci-Zahlen. Der Zähler muss
vor jedem Funktionsaufruf wieder zurück gesetzt werden. So lassen sich
nun die Zahlen untersuchen

    \begin{Verbatim}[commandchars=\\\{\}]
{\color{incolor}In [{\color{incolor}3}]:} \PY{c+c1}{\PYZsh{}Funktion welche die ersten 20 Fibonacci Zahlen plotet}
        
        \PY{k}{for} \PY{n}{i} \PY{o+ow}{in} \PY{n+nb}{range}\PY{p}{(}\PY{l+m+mi}{20}\PY{p}{)}\PY{p}{:}
            \PY{n}{count} \PY{o}{=} \PY{l+m+mi}{0}    
            \PY{n+nb}{print}\PY{p}{(}\PY{n}{f}\PY{l+s+s2}{\PYZdq{}}\PY{l+s+s2}{Position n=}\PY{l+s+si}{\PYZob{}i\PYZcb{}}\PY{l+s+s2}{, Fibonacci number: }\PY{l+s+s2}{\PYZob{}}\PY{l+s+s2}{fib(i)\PYZcb{}, calls }\PY{l+s+si}{\PYZob{}count\PYZcb{}}\PY{l+s+s2}{\PYZdq{}}\PY{p}{)}
           
            
\end{Verbatim}


    \begin{Verbatim}[commandchars=\\\{\}]
Position n=0, Fibonacci number: 0, calls 1
Position n=1, Fibonacci number: 1, calls 1
Position n=2, Fibonacci number: 1, calls 3
Position n=3, Fibonacci number: 2, calls 5
Position n=4, Fibonacci number: 3, calls 9
Position n=5, Fibonacci number: 5, calls 15
Position n=6, Fibonacci number: 8, calls 25
Position n=7, Fibonacci number: 13, calls 41
Position n=8, Fibonacci number: 21, calls 67
Position n=9, Fibonacci number: 34, calls 109
Position n=10, Fibonacci number: 55, calls 177
Position n=11, Fibonacci number: 89, calls 287
Position n=12, Fibonacci number: 144, calls 465
Position n=13, Fibonacci number: 233, calls 753
Position n=14, Fibonacci number: 377, calls 1219
Position n=15, Fibonacci number: 610, calls 1973
Position n=16, Fibonacci number: 987, calls 3193
Position n=17, Fibonacci number: 1597, calls 5167
Position n=18, Fibonacci number: 2584, calls 8361
Position n=19, Fibonacci number: 4181, calls 13529

    \end{Verbatim}

    \subsection{4. und 5. Zeit und
Effizienz}\label{und-5.-zeit-und-effizienz}

Hier ist eine Klasse welche die Fibonacci-Zahlen iterativ und rekursiv
berechnet und dabei die Zeit misst.

    \begin{Verbatim}[commandchars=\\\{\}]
{\color{incolor}In [{\color{incolor}4}]:} \PY{c+c1}{\PYZsh{}import fast timer}
        \PY{k+kn}{from} \PY{n+nn}{timeit} \PY{k}{import} \PY{n}{default\PYZus{}timer} \PY{k}{as} \PY{n}{timer}
        
        \PY{k+kn}{import} \PY{n+nn}{sys}
        \PY{n}{sys}\PY{o}{.}\PY{n}{setrecursionlimit}\PY{p}{(}\PY{l+m+mi}{10000}\PY{p}{)}
        
        \PY{c+c1}{\PYZsh{}Get Input from user not implemented}
        \PY{c+c1}{\PYZsh{}set the n\PYZhy{}the fibonaccinumber you would like to get.}
        \PY{n}{number} \PY{o}{=} \PY{l+m+mi}{4}
        
        \PY{c+c1}{\PYZsh{} 1. Implementation of two functions which calculate the n\PYZhy{}the Fibonacci number. One iterative the other recursive.}
        \PY{c+c1}{\PYZsh{} class to calculate the fibonacci numbers}
        \PY{k}{class} \PY{n+nc}{Fibonacci}\PY{p}{:}
            
            \PY{n}{\PYZus{}usedTime} \PY{o}{=} \PY{l+m+mi}{0}
        
            \PY{k}{def} \PY{n+nf}{getNTheFibonacciNumberRecursive}\PY{p}{(}\PY{n+nb+bp}{self}\PY{p}{,} \PY{n}{n}\PY{p}{)}\PY{p}{:}
                \PY{n}{start} \PY{o}{=} \PY{n}{timer}\PY{p}{(}\PY{p}{)}
                \PY{n}{result} \PY{o}{=} \PY{n+nb+bp}{self}\PY{o}{.}\PY{n}{\PYZus{}fibonacciRecursive}\PY{p}{(}\PY{n}{n}\PY{p}{)}
                \PY{n}{stop} \PY{o}{=} \PY{n}{timer}\PY{p}{(}\PY{p}{)}
                \PY{n}{fib}\PY{o}{.}\PY{n}{calcTime}\PY{p}{(}\PY{n}{start}\PY{p}{,}\PY{n}{stop}\PY{p}{)}
                \PY{k}{return} \PY{n}{result}
        
            \PY{c+c1}{\PYZsh{} Get the n\PYZhy{}the fibonaccinumber iterativ}
            \PY{k}{def} \PY{n+nf}{getNTheFibonacciNumberIterative}\PY{p}{(}\PY{n+nb+bp}{self}\PY{p}{,} \PY{n}{n}\PY{p}{)}\PY{p}{:}
                \PY{n}{start} \PY{o}{=} \PY{n}{timer}\PY{p}{(}\PY{p}{)}
                
                \PY{n}{nextterm} \PY{o}{=} \PY{l+m+mi}{0}
                \PY{n}{present} \PY{o}{=} \PY{l+m+mi}{1}
                \PY{n}{previous} \PY{o}{=} \PY{l+m+mi}{0}
        
                \PY{c+c1}{\PYZsh{} Get only values greater than 0}
                \PY{k}{if} \PY{n}{n} \PY{o}{\PYZlt{}} \PY{l+m+mi}{0}\PY{p}{:}
                    \PY{n+nb}{print}\PY{p}{(}\PY{l+s+s2}{\PYZdq{}}\PY{l+s+s2}{Incorrect input}\PY{l+s+s2}{\PYZdq{}}\PY{p}{)}
        
                \PY{n}{i} \PY{o}{=} \PY{l+m+mi}{0}
                \PY{k}{while} \PY{n}{i} \PY{o}{\PYZlt{}} \PY{n}{n}\PY{p}{:}
                    \PY{n}{nextterm} \PY{o}{=} \PY{n}{present} \PY{o}{+} \PY{n}{previous}
                    \PY{n}{present} \PY{o}{=} \PY{n}{previous}
                    \PY{n}{previous} \PY{o}{=} \PY{n}{nextterm}
                    \PY{n}{i} \PY{o}{+}\PY{o}{=} \PY{l+m+mi}{1} 
        
                \PY{n}{stop} \PY{o}{=} \PY{n}{timer}\PY{p}{(}\PY{p}{)}
                \PY{n+nb+bp}{self}\PY{o}{.}\PY{n}{calcTime}\PY{p}{(}\PY{n}{start}\PY{p}{,}\PY{n}{stop}\PY{p}{)}
                \PY{k}{return} \PY{n}{nextterm}
        
            \PY{k}{def} \PY{n+nf}{calcTime}\PY{p}{(}\PY{n+nb+bp}{self}\PY{p}{,} \PY{n}{startTime}\PY{p}{,} \PY{n}{stopTime}\PY{p}{)}\PY{p}{:}
                \PY{n+nb+bp}{self}\PY{o}{.}\PY{n}{\PYZus{}usedTime} \PY{o}{=} \PY{p}{(}\PY{n}{stopTime}\PY{o}{\PYZhy{}}\PY{n}{startTime}\PY{p}{)}
        
            \PY{k}{def} \PY{n+nf}{getUsedTimeToCalculateNtheNumber}\PY{p}{(}\PY{n+nb+bp}{self}\PY{p}{)}\PY{p}{:}
                \PY{k}{return} \PY{n+nb+bp}{self}\PY{o}{.}\PY{n}{\PYZus{}usedTime}
        
            \PY{c+c1}{\PYZsh{} private get the n\PYZhy{}the fibonaccinumber recursive}
            \PY{k}{def} \PY{n+nf}{\PYZus{}fibonacciRecursive}\PY{p}{(}\PY{n+nb+bp}{self}\PY{p}{,} \PY{n}{n}\PY{p}{)}\PY{p}{:}
                \PY{n}{retVal} \PY{o}{=} \PY{l+m+mi}{0}
                \PY{c+c1}{\PYZsh{} Get only values greater than 0}
                \PY{k}{if} \PY{n}{n} \PY{o}{\PYZlt{}} \PY{l+m+mi}{0}\PY{p}{:}
                    \PY{n+nb}{print}\PY{p}{(}\PY{l+s+s2}{\PYZdq{}}\PY{l+s+s2}{Incorrect input, n has to be greater than 0}\PY{l+s+s2}{\PYZdq{}}\PY{p}{)}           
                \PY{c+c1}{\PYZsh{} First Fibonacci number is 0}
                \PY{k}{elif} \PY{n}{n} \PY{o}{==} \PY{l+m+mi}{0}\PY{p}{:}
                    \PY{n}{retVal} \PY{o}{=} \PY{l+m+mi}{0}
                \PY{c+c1}{\PYZsh{} Second Fibonacci number is 1}
                \PY{k}{elif} \PY{n}{n} \PY{o}{==} \PY{l+m+mi}{1}\PY{p}{:}
                    \PY{n}{retVal} \PY{o}{=} \PY{l+m+mi}{1}
                \PY{k}{else}\PY{p}{:}
                    \PY{n}{retVal} \PY{o}{=} \PY{n+nb+bp}{self}\PY{o}{.}\PY{n}{\PYZus{}fibonacciRecursive}\PY{p}{(}\PY{n}{n} \PY{o}{\PYZhy{}} \PY{l+m+mi}{1}\PY{p}{)} \PY{o}{+} \PY{n+nb+bp}{self}\PY{o}{.}\PY{n}{\PYZus{}fibonacciRecursive}\PY{p}{(}\PY{n}{n} \PY{o}{\PYZhy{}} \PY{l+m+mi}{2}\PY{p}{)}
                \PY{k}{return} \PY{n}{retVal}
        
        
        \PY{c+c1}{\PYZsh{} create instance of class}
        \PY{n}{fib} \PY{o}{=} \PY{n}{Fibonacci}\PY{p}{(}\PY{p}{)}
        
        
        
        \PY{c+c1}{\PYZsh{}At position 10 is Fibonacci number 100}
        \PY{n+nb}{print}\PY{p}{(}\PY{l+s+s2}{\PYZdq{}}\PY{l+s+se}{\PYZbs{}n}\PY{l+s+s2}{\PYZdq{}}\PY{p}{)}
        \PY{n+nb}{print}\PY{p}{(}\PY{l+s+s2}{\PYZdq{}}\PY{l+s+s2}{Recursive calculation of the Fibonacci Number}\PY{l+s+s2}{\PYZdq{}}\PY{p}{)}
        \PY{n+nb}{print}\PY{p}{(}\PY{n}{f}\PY{l+s+s2}{\PYZdq{}}\PY{l+s+s2}{At position }\PY{l+s+si}{\PYZob{}number\PYZcb{}}\PY{l+s+s2}{ is Fibonacci number: }\PY{l+s+s2}{\PYZob{}}\PY{l+s+s2}{fib.getNTheFibonacciNumberRecursive(number)\PYZcb{} }\PY{l+s+se}{\PYZbs{}n}\PY{l+s+s2}{Process time: }\PY{l+s+s2}{\PYZob{}}\PY{l+s+s2}{fib.getUsedTimeToCalculateNtheNumber()*1000\PYZcb{}ms}\PY{l+s+s2}{\PYZdq{}}\PY{p}{)}
        
        \PY{n+nb}{print}\PY{p}{(}\PY{l+s+s2}{\PYZdq{}}\PY{l+s+se}{\PYZbs{}n}\PY{l+s+s2}{\PYZdq{}}\PY{p}{)}
        \PY{n+nb}{print}\PY{p}{(}\PY{l+s+s2}{\PYZdq{}}\PY{l+s+s2}{Iterative calculation of the Fibonacci Number:}\PY{l+s+s2}{\PYZdq{}}\PY{p}{)}
        \PY{n+nb}{print}\PY{p}{(}\PY{n}{f}\PY{l+s+s2}{\PYZdq{}}\PY{l+s+s2}{At position }\PY{l+s+si}{\PYZob{}number\PYZcb{}}\PY{l+s+s2}{ is Fibonacci number: }\PY{l+s+s2}{\PYZob{}}\PY{l+s+s2}{fib.getNTheFibonacciNumberIterative(number)\PYZcb{}}\PY{l+s+s2}{\PYZdq{}}\PY{p}{)}
        \PY{n+nb}{print}\PY{p}{(}\PY{n}{f}\PY{l+s+s2}{\PYZdq{}}\PY{l+s+s2}{Process time: }\PY{l+s+s2}{\PYZob{}}\PY{l+s+s2}{fib.getUsedTimeToCalculateNtheNumber()*1000\PYZcb{}ms}\PY{l+s+s2}{\PYZdq{}}\PY{p}{)}
\end{Verbatim}


    \begin{Verbatim}[commandchars=\\\{\}]


Recursive calculation of the Fibonacci Number
At position 4 is Fibonacci number: 3 
Process time: 0.0026879997676587664ms


Iterative calculation of the Fibonacci Number:
At position 4 is Fibonacci number: 3
Process time: 0.0010710000424296595ms

    \end{Verbatim}

    \subsection{Diskussion der Ergebnisse}\label{diskussion-der-ergebnisse}

\subsubsection{Aufgabe 1 und 2}\label{aufgabe-1-und-2}

Wie wir im Plot bei Aufgabe 3 sehen stimmen die Rückgabewerte der
funktion fib() für die Startwerte \$ f\_0:=0 \text{ und } f\_1:=1 \$
sowie für die weiteren Werte \(f_n\).

\subsubsection{Aufgabe 3}\label{aufgabe-3}

Anhand der geploteten zahlen sehen wir, dass die Anzahl Funktionsaufrufe
immer etwa 3x grösser ist als die Fibonacci Zahl. Daher gehen wir mit
einem "educated guess" vor und nehmen an es ist eine Folge welche
ähnlich der Fibonacci-Folge ist, also ähnlich wie \$
a\_n:=a\_\{n−1\}+a\_\{n−2\} \$

Testen wir dies für \$ a\_2 = a\_0 + a\_1 a\_2 = 1 + 1 = 2 \$ gemäss
unseres plott sollte aber \(a_2 = 3\) sein.

Wir passen die Formel an zu \$ a\_n:=a\_\{n−1\}+a\_\{n−2\} + 1 \$

Nun is \$ a2 = 3 \$

Testen wir \$ a3 = a2 + a1 + 1 = 3 + 1 + 1 = 5 \$ dies entspricht
unseren geplotteten Daten.

Wenden wir die Theorie der Induktion an. Die ersten zwei start Werte
sind explixit definiert. Daher definieren wir auch bei der Formel für
die Anzahl aufrufe. Wir sehen sehr deutlich, dass die itterative Methode
wesentlich kürzere Rechenzeiten beansprucht und wesentlich efizienter
ist. Dies verwundert nicht da die Anzahl Funktionsaufrufe bei der
rekursiven Methode exponentiell grösser wird im Gegensatz zur
itterativen Methode.

Gehen wir mit der Aussage \(a_n:=a_{n−1}+a_{n−2}+1\) den
Induktionsalgorithmus (1) durch.

\begin{enumerate}
\def\labelenumi{\arabic{enumi}.}
\item
  Induktionsanfang \$ a\_0 = 1 \textbackslash{} a\_1 = 1 \$
\item
  Induktionsvoraussetzung \$ a\_2 = 3 \$ Die Aussage gilt für ein \$ n
  \in \mathbb{N} \$ n 0 .
\item
  Induktionsbehauptung Die Aussage gilt für \(n\) als auch für \(n+1\)
\item
  Induktionsschluss:
\end{enumerate}

\$ a\_\{n+1\} = a\_\{n+1-1\} + a\_\{n+1-2\} + 1 \$

\$ a\_\{n+1\} = a\_n + a\_\{n-1\} + 1
\Longleftrightarrow a\_n:=a\_\{n−1\}+a\_\{n−2\}+1 \square \$

\$ a\_3 = a\_2 + a\_1 +1 \$

Quelle:\\
(1) Vollsändige Induktion Dalwigk F. ISBN 978-3-662-58632-7

\subsubsection{Aufgabe 4 und 5}\label{aufgabe-4-und-5}

Wir sehen sehr deutlich, dass die itterative Methode wesentlich kürzere
Rechenzeiten beansprucht und wesentlich efizienter ist. Dies verwundert
nicht da die Anzahl Funktionsaufrufe bei der rekursiven Methode
exponentiell grösser wird im Gegensatz zur itterativen Methode bei
welcher die Rechenzeit linear länger wird.


    % Add a bibliography block to the postdoc
    
    
    
    \end{document}
